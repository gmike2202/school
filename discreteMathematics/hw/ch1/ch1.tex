% --------------------------------------------------------------
% This is all preamble stuff that you don't have to worry about.
% Head down to where it says "Start here"
% --------------------------------------------------------------
 
\documentclass[12pt]{article}
 
\usepackage[margin=1in]{geometry} 
\usepackage{amsmath,amsthm,amssymb}
 
\newcommand{\N}{\mathbb{N}}
\newcommand{\Z}{\mathbb{Z}}
\newcommand{\W}{\mathbb{W}}
 
\newenvironment{theorem}[2][Theorem]{\begin{trivlist}
\item[\hskip \labelsep {\bfseries #1}\hskip \labelsep {\bfseries #2.}]}{\end{trivlist}}
\newenvironment{lemma}[2][Lemma]{\begin{trivlist}
\item[\hskip \labelsep {\bfseries #1}\hskip \labelsep {\bfseries #2.}]}{\end{trivlist}}
\newenvironment{exercise}[2][Exercise]{\begin{trivlist}
\item[\hskip \labelsep {\bfseries #1}\hskip \labelsep {\bfseries #2.}]}{\end{trivlist}}
\newenvironment{reflection}[2][Reflection]{\begin{trivlist}
\item[\hskip \labelsep {\bfseries #1}\hskip \labelsep {\bfseries #2.}]}{\end{trivlist}}
\newenvironment{proposition}[2][Proposition]{\begin{trivlist}
\item[\hskip \labelsep {\bfseries #1}\hskip \labelsep {\bfseries #2.}]}{\end{trivlist}}
\newenvironment{corollary}[2][Corollary]{\begin{trivlist}
\item[\hskip \labelsep {\bfseries #1}\hskip \labelsep {\bfseries #2.}]}{\end{trivlist}}
\newenvironment{homework}[2][Homework]{\begin{trivlist}
\item[\hskip \labelsep {\bfseries #1}\hskip \labelsep {\bfseries #2.}]}{\end{trivlist}}
\newenvironment{other}[2][]{\begin{trivlist}
\item[\hskip \labelsep {\bfseries #1}\hskip \labelsep {\bfseries #2.}]}{\end{trivlist}}
 
\begin{document}
 
% --------------------------------------------------------------
%                         Start here
% --------------------------------------------------------------
 
%\renewcommand{\qedsymbol}{\filledbox}
 
\title{Chapter 1}%replace X with the appropriate homework chapter or title
\author{Michael Garcia\\ %replace with your name
MAT 203 - Discrete Mathematics} 
 
\maketitle
 
\begin{theorem}{4} %You can use theorem, proposition, homework, exercise, or reflection here.  Modify x.yz to be whatever number you are proving or solving
  The sum of two odd numbers is even.
\end{theorem}
 
\begin{proof}
  Recall the definition of an odd number to be a number, n, that can be written as 2k-1 where k is an integer. 
  Recall also that the definition of an even number is a number, m, that can be written as 2j where j is an integer.
  Let $n_1 = |2k-1|$ and $n_2 = |2j-1|$. By adding these two we get $2k-1+2j-1$ which turns into $2k+2j-2$. By factoring out the 2 we receive $2(k+j-1)$.
\end{proof}

\begin{other}{Result}
 Thus, since k and j are integers, subtracting 1 still results in an integer. Therefore we can rewrite $2k+j-1$ as m. 
 Therefore, we are left with the definition of an even number, ergo 2m.
\end{other}

\begin{theorem}{7}
 Every binary number n that ends in 0 is even 
\end{theorem}

\begin{proof}
The definition of a binary number is a number that can be written as\\
$$n = a_{n-1} \cdot{2^{n-1}} +...+ a_1 \cdot{2^1} + a_0 \cdot{2^0}$$
Each number except the last one can have a 2 factored out. Therefore let $m = a_{n-1} \cdot{2^{n-1}} +...+ a_1 \cdot{2^1}$.
As such, the formula can be written as \\$$n = 2m + a_0 \cdot{2^0}$$.\\
Recall that the definition of an even number to be a number that can be written as 2m.
Since $2^0 = 1$, and since the last number is 0, we have \\$$n = 2m + 0 \cdot{1} = 2m$$.
\end{proof}

\begin{other}{Result}
  Thus, through the definition of an even number, any integer that can be written as 2m, where m is an integer is even. 
  Therefore since any binary number n ending in 0 can be written as 2m, it follows that n is even.
\end{other}

\begin{theorem}{10}% Prove or find a counterxample
 The difference of two consecutive perfect squares is odd
\end{theorem}
\begin{proof}

\end{proof}
\begin{other}{Result}

\end{other}

\begin{theorem}{17}
  At least two flights at Chicaog O'Hare International AIrport must take off within 90 seconds of each other
\end{theorem}
\begin{proof}

\end{proof}
\begin{other}{Result}

\end{other}

\begin{theorem}{22a}%!
  
\end{theorem}
\begin{proof}

\end{proof}
\begin{other}{Result}

\end{other}

\begin{theorem}{22b}%!

\end{theorem}
\begin{proof}

\end{proof}
\begin{other}{Result}

\end{other}

\begin{theorem}{23}%Suggestion: do one case for n odd and one case for n even.
  If n is any integer, then $3n^3 + n + 5$ is odd.
\end{theorem}
\begin{proof}

\end{proof}
\begin{other}{Result}

\end{other}
% --------------------------------------------------------------
%     You don't have to mess with anything below this line.
% --------------------------------------------------------------
 
\end{document}
