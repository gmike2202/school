% --------------------------------------------------------------
% This is all preamble stuff that you don't have to worry about.
% Head down to where it says "Start here"
% --------------------------------------------------------------
 
\documentclass[12pt]{article}
 
\usepackage[margin=1in]{geometry} 
\usepackage{amsmath,amsthm,amssymb}
 
\newcommand{\N}{\mathbb{N}}
\newcommand{\Z}{\mathbb{Z}}
\newcommand{\W}{\mathbb{W}}
 
\newenvironment{theorem}[2][Theorem]{\begin{trivlist}
\item[\hskip \labelsep {\bfseries #1}\hskip \labelsep {\bfseries #2.}]}{\end{trivlist}}
\newenvironment{lemma}[2][Lemma]{\begin{trivlist}
\item[\hskip \labelsep {\bfseries #1}\hskip \labelsep {\bfseries #2.}]}{\end{trivlist}}
\newenvironment{exercise}[2][Exercise]{\begin{trivlist}
\item[\hskip \labelsep {\bfseries #1}\hskip \labelsep {\bfseries #2.}]}{\end{trivlist}}
\newenvironment{reflection}[2][Reflection]{\begin{trivlist}
\item[\hskip \labelsep {\bfseries #1}\hskip \labelsep {\bfseries #2.}]}{\end{trivlist}}
\newenvironment{proposition}[2][Proposition]{\begin{trivlist}
\item[\hskip \labelsep {\bfseries #1}\hskip \labelsep {\bfseries #2.}]}{\end{trivlist}}
\newenvironment{corollary}[2][Corollary]{\begin{trivlist}
\item[\hskip \labelsep {\bfseries #1}\hskip \labelsep {\bfseries #2.}]}{\end{trivlist}}
\newenvironment{homework}[2][Homework]{\begin{trivlist}
\item[\hskip \labelsep {\bfseries #1}\hskip \labelsep {\bfseries #2.}]}{\end{trivlist}}
\newenvironment{other}[2][]{\begin{trivlist}
\item[\hskip \labelsep {\bfseries #1}\hskip \labelsep {\bfseries #2.}]}{\end{trivlist}}
 
\begin{document}
 
% --------------------------------------------------------------
%                         Start here
% --------------------------------------------------------------
 
%\renewcommand{\qedsymbol}{\filledbox}
 
\title{Chapter 2}%replace X with the appropriate homework chapter or title
\author{Michael Garcia\\ %replace with your name
MAT 203 - Discrete Mathematics} 
 
\maketitle
 
\begin{theorem}{14} %You can use theorem, proposition, homework, exercise, or reflection here.  Modify x.yz to be whatever number you are proving or solving
 It is not true that $(AXB)\cup (CXD) = (A\cup C) X (B\cup D)$.
\end{theorem}
 
\begin{proof}
  Let 
  \begin{align*}
    A &= {1}\\
    B &= {2}\\
    C &= {3}\\
    D &= {4}\\
  \end{align*}
 Then,
 \begin{align*}
   A \times B &= \{(1,2)\}\\
   C \times D &= \{(3,4)\}\\
   A \cup C &= \{1,3\}\\
   B \cup D &= \{2,4\}\\
   (A \times B) \cup (C \times D) &= \{(1,2),(3,4)\}\\
   (A \cup B) \times (B \cup D)\\ &= \{(1,2), (1,4), (3,2), (3,4)\}\\
 \end{align*}
\end{proof}

\begin{other}{result}
  As $(A \times B) \cup (C \times D)$ and $(A \cup B) \times (B \cup D)$ produce different sets, they cannot be said to be equal.
\end{other}

\begin{exercise}{18}
  Write this in English:
   $$\forall k\in 3\mathbb{Z},\exists S\subseteq \mathbb{N},| S |= k$$  
  (Is it true?) What is the negation of this statement? (Is the negation true?)
\end{exercise}

\begin{other}{result}
  For all integers $k$ that are multiples of 3, there exists a subset $S$ of the natural numbers, so that the cardinality is equal to $k$.\\
  The statement is not true, as there cannot be a set with a negative number of elements.\\
  Negation in logic notation: $$\exists k \in 3 \mathbb{Z}, \forall S \subseteq \mathbb{N}, |S| \neq K$$\\
  Negation in English: There exists an integer $k$ that is a multiple of 3, for all subsets $S$ of the natural numbers, so that the cardinality of $S$ is not equal to $k$.\\
  The statement is true, as there cannot be a susbet with a negative number of elements.
\end{other}

\begin{theorem}{20}
$\mathbb{Z} = \{3k|k\in \mathbb{Z}\}\cup\{3k+1|k\in \mathbb{Z}\}\cup\{3k+2|k\in \mathbb{Z}\}$
\end{theorem}

\begin{proof}
  Recall the definition of a union between sets to include all the elements from each set.\\
  Recall the closure property to state that addition and multiplication between integers results in an integer.\\
  Let $S = \{3k|k\in \mathbb{Z}\}\cup\{3k+1|k\in \mathbb{Z}\}\cup\{3k+2|k\in \mathbb{Z}\}$.\\
  $3k$ is an integer for $k\in \mathbb{Z}$. Therefore, $3k+1$, $3k+2 \in \mathbb{Z}$ through closure property. \\
  Thus, $S\subseteq \mathbb{Z}$.\\

  Using the division algorithm, we get $$n = 3q + r \text{ where } 0 \leq r < 3$$\\
  If r = 0, then n = 3q, which belongs to $3k| k \in \mathbb{Z}$\\
  If r = 1, then n = 3q + 1, which belongs to $3k + 1| k \in \mathbb{Z}$\\
  If r = 2, then n = 3q + 2, which belongs to $3k + 2| k \in \mathbb{Z}$\\
  This shows that every integer $n$ fits into one of the three sets. Thus, $\mathbb{Z} \subseteq S$
\end{proof}

\begin{other}{result}
  Hence, since $\mathbb{Z}$ and $S$ both contain the same elements, it can be said that $$Z = \{3k|k\in \mathbb{Z}\}\cup\{3k+1|k\in \mathbb{Z}\}\cup\{3k+2|k\in \mathbb{Z}\}$$\\

\end{other}

\begin{exercise}{29}
Write the negation of $x\text{ is prime or }x < 52$. (Don't say, "It's not true that ...").
\end{exercise}

\begin{other}{negation}
  $x\text{ is not prime and }x\geq 52$.
\end{other}

\begin{exercise}{32}
Write each of the following statements using formal logic notation.\\
(a) Even numbers are never prime\\
(b) Triangles never have four sides.\\
(c) There are no integers $a,b$ such that $a^2/b^2 = 2$\\
(d) No square number immediately follows a prime number.\\
\end{exercise}

\begin{other}{translations}
Below are the translations.\\
(a) $$\forall n \in \mathbb{Z}, \forall p \in \mathbb{P}, \neg (2n = p)$$\\
(b) Let $T = \text{Triangle}$ and $S = \text{Number of sides a shape has.}$\\
Then, 
$$
\forall T, \neg (S = 4)
$$\\
(c) $$~ \exists a, b \in \mathbb{Z} | a^2 / b^2 = 2$$\\
(d) $$\neg ( \exists p \in \mathbb{P}, \exists n \in \mathbb{Z} | n^2 = p+1)$$\\
\end{other}

\begin{theorem}{40}
  $A\cap B = A\setminus B$
\end{theorem}

\begin{proof}[disproof]
  Recall the definition of an intersection between sets to contain elements common to each set.\\
  Recall subtraction between sets to mean creating a set containing only elements from the first set that are not present in the second set.\\
  Let $A = \{1, 2\} \text{ and } B=\{2, 3\}$.\\
  Then,
  \begin{align*}
    A \cap B &= \{2\}\\
    A\setminus{B} &= \{1\}\\
  \end{align*}
  Thus, since ${2} \ne {1}$, then $A\cap B \neq A\setminus B$.
\end{proof}

\begin{other}{Condition for Truth}
  Let $A = \emptyset$. For any set B, we have
  \begin{align*}
    A \cap B &= \emptyset \\
    A\setminus{B} &= \emptyset\\ 
  \end{align*}
 Thus, $A\cap B = A\setminus B$ when $A = \emptyset$
\end{other}

\begin{other}{result}
  Thus, the only case in which the statement is true is when $A = \emptyset$ for any set B.
\end{other}

% --------------------------------------------------------------
%     You don't have to mess with anything below this line.
% --------------------------------------------------------------
 
\end{document}
