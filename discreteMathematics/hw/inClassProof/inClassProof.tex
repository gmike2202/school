
% --------------------------------------------------------------
% This is all preamble stuff that you don't have to worry about.
% Head down to where it says "Start here"
% --------------------------------------------------------------
 
\documentclass[12pt]{article}
 
\usepackage[margin=1in]{geometry} 
\usepackage{amsmath,amsthm,amssymb}
 
\newcommand{\N}{\mathbb{N}}
\newcommand{\Z}{\mathbb{Z}}
\newcommand{\W}{\mathbb{W}}
 
\newenvironment{theorem}[2][Theorem]{\begin{trivlist}
\item[\hskip \labelsep {\bfseries #1}\hskip \labelsep {\bfseries #2.}]}{\end{trivlist}}
\newenvironment{lemma}[2][Lemma]{\begin{trivlist}
\item[\hskip \labelsep {\bfseries #1}\hskip \labelsep {\bfseries #2.}]}{\end{trivlist}}
\newenvironment{exercise}[2][Exercise]{\begin{trivlist}
\item[\hskip \labelsep {\bfseries #1}\hskip \labelsep {\bfseries #2.}]}{\end{trivlist}}
\newenvironment{reflection}[2][Reflection]{\begin{trivlist}
\item[\hskip \labelsep {\bfseries #1}\hskip \labelsep {\bfseries #2.}]}{\end{trivlist}}
\newenvironment{proposition}[2][Proposition]{\begin{trivlist}
\item[\hskip \labelsep {\bfseries #1}\hskip \labelsep {\bfseries #2.}]}{\end{trivlist}}
\newenvironment{corollary}[2][Corollary]{\begin{trivlist}
\item[\hskip \labelsep {\bfseries #1}\hskip \labelsep {\bfseries #2.}]}{\end{trivlist}}
\newenvironment{homework}[2][Homework]{\begin{trivlist}
\item[\hskip \labelsep {\bfseries #1}\hskip \labelsep {\bfseries #2.}]}{\end{trivlist}}
\newenvironment{other}[2][]{\begin{trivlist}
\item[\hskip \labelsep {\bfseries #1}\hskip \labelsep {\bfseries #2.}]}{\end{trivlist}}
 
\begin{document}
 
% --------------------------------------------------------------
%                         Start here
% --------------------------------------------------------------
 
%\renewcommand{\qedsymbol}{\filledbox}
 
\title{Change Title}%replace X with the appropriate homework chapter or title
\author{Michael Garcia, Ayden Barclay, Ty Turoczy, Harlow Sharp\\ %replace with your name
MAT 203 - Discrete Mathematics} 
 
\maketitle
 
\begin{theorem}\\   %You can use theorem, proposition, homework, exercise, or reflection here.  Modify x.yz to be whatever number you are proving or solving
  The function $f: \mathbb{Z} \rightarrow \mathbb{N}$ defined by
  $f(n) = 
  \begin{cases}
    2n & \text{ if } n> 0\\
    -2n+1 & \text{ if } n\leq 0\\
  \end{cases}$
  is an injection.
\end{theorem}
 
\begin{proof}
  Recall that to prove that a function is an injection, we must prove that whenever $f(a) = f(b)$, then it must be true that $a=b$.\\
  In other words, we must show that the preimage of every element in the range is associated with exactly one element of the domain.\\
  Let $a, b \in \mathbb{Z} \text{ and assume that } f(a) = f(b)$.\\
  \\
  Notice that whenever the function input, n, is greater than zero, then the output, $f(n) = 2n$, is even.\\
  \\
  Similarly, whenever the function input, n, is less than or equal to zero, then the output, $f(n) = -2n+1$, is odd.\\
  \\
  An even number cannot equal an odd number, so there are two possible cases for $f(a)=f(b)$; either both a and b are greater than zero, so that $f(a)$ and $f(b)$ are both even, or both a and b are less than or equal to zero, so that $f(a)$ and $f(b)$ are both odd.\\
  \\
  \textbf{Case 1:}\\
  Assume both $a$ and $b$ are greater than zero.\\
  Then $f(a) = 2a \text{ and }f(b)= 2b$\\
  By our assumption that $f(a)=f(b)$, we get:
  \begin{align*}
    f(a) &= f(b)\\
    2(a) &= 2(b)\\
    a &= b\\
  \end{align*}
  \\
  \textbf{Case 2:}\\
  Assume both $a$ and $b$ are less than or equal to zero.\\
  Then $f(a) = -2a+1 \text{ and }f(b)= -2b+1$\\
  By our assumption that $f(a)=f(b)$, we get:
  \begin{align*}
  f(a) &= f(b)\\
  -2(a) + 1 &= -2(b) + 1\\
  -2(a) &= -2(b)\\
  a &= b\\
  \end{align*}
  \\
  In all possible cases, whenever $f(a) = f(b)$, we get $a=b$.\\
\\
  This is what we needed to show\\
\\
  Thus, the function is an injection.
\end{proof}

% --------------------------------------------------------------
%     You don't have to mess with anything below this line.
% --------------------------------------------------------------
 
\end{document}


