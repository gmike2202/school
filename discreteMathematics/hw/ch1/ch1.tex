% --------------------------------------------------------------
% This is all preamble stuff that you don't have to worry about.
% Head down to where it says "Start here"
% --------------------------------------------------------------
 
\documentclass[12pt]{article}
 
\usepackage[margin=1in]{geometry} 
\usepackage{amsmath,amsthm,amssymb}
 
\newcommand{\N}{\mathbb{N}}
\newcommand{\Z}{\mathbb{Z}}
\newcommand{\W}{\mathbb{W}}
 
\newenvironment{theorem}[2][Theorem]{\begin{trivlist}
\item[\hskip \labelsep {\bfseries #1}\hskip \labelsep {\bfseries #2.}]}{\end{trivlist}}
\newenvironment{lemma}[2][Lemma]{\begin{trivlist}
\item[\hskip \labelsep {\bfseries #1}\hskip \labelsep {\bfseries #2.}]}{\end{trivlist}}
\newenvironment{exercise}[2][Exercise]{\begin{trivlist}
\item[\hskip \labelsep {\bfseries #1}\hskip \labelsep {\bfseries #2.}]}{\end{trivlist}}
\newenvironment{reflection}[2][Reflection]{\begin{trivlist}
\item[\hskip \labelsep {\bfseries #1}\hskip \labelsep {\bfseries #2.}]}{\end{trivlist}}
\newenvironment{proposition}[2][Proposition]{\begin{trivlist}
\item[\hskip \labelsep {\bfseries #1}\hskip \labelsep {\bfseries #2.}]}{\end{trivlist}}
\newenvironment{corollary}[2][Corollary]{\begin{trivlist}
\item[\hskip \labelsep {\bfseries #1}\hskip \labelsep {\bfseries #2.}]}{\end{trivlist}}
\newenvironment{homework}[2][Homework]{\begin{trivlist}
\item[\hskip \labelsep {\bfseries #1}\hskip \labelsep {\bfseries #2.}]}{\end{trivlist}}
\newenvironment{other}[2][]{\begin{trivlist}
\item[\hskip \labelsep {\bfseries #1}\hskip \labelsep {\bfseries #2.}]}{\end{trivlist}}
 
\begin{document}
 
% --------------------------------------------------------------
%                         Start here
% --------------------------------------------------------------
 
%\renewcommand{\qedsymbol}{\filledbox}
 
\title{Chapter 1}%replace X with the appropriate homework chapter or title
\author{Michael Garcia\\ %replace with your name
MAT 203 - Discrete Mathematics} 
 
\maketitle
 
\begin{theorem}{4} %You can use theorem, proposition, homework, exercise, or reflection here.  Modify x.yz to be whatever number you are proving or solving
  The sum of two odd numbers is even.
\end{theorem}
 
\begin{proof}
  Recall the definition of an odd number to be a number, $n$, that can be written as $2k-1$ where $k$ is an integer. 
  Recall also that the definition of an even number is a number that can be written as $2m$ where $m$ is an integer.\\
  Let $n_1$ and $n_2$ both be odd numbers.
  Then,
  \begin{align*}
    n_1 + n_2 &=(2k-1) + (2j-1)\\
               &= 2k-1+2j-1\\
               &= 2k+2j-2\\
               &= 2(k+j-1)\\
               &= 2m
  \end{align*}
  where $m = k + j - 1$. As integers are closed under addition and subtraction $m$ itself must be an integer.
\end{proof}

\begin{other}{Result}
 Therefore, $k + j - 1$ can be written as $m$, and we are left with $2m$. 
\end{other}

\begin{theorem}{7}
 Every binary number, $n$, that ends in $0$ is even.
\end{theorem}

\begin{proof}
Recall the definition of an even number to be a number that can be written as $2m$. The definition of a binary number is a number that can be written as $$n = a_{n-1} \cdot{2^{n-1}} +...+ a_1 \cdot{2^1} + a_0 \cdot{2^0}$$
Let $m = a_{n-1} \cdot{2^{n-1}} +...+ a_1 \cdot{2^1}$\\
Thus, we have $$n = 2m + 0 \cdot{1} = 2m$$.
\end{proof}

\begin{other}{Result}
  Thus, through substitution, any binary number $n$ ending in $0$ can be written as $2m$.
\end{other}
\pagebreak
\begin{theorem}{10}% Prove or find a counterxample
 The difference of two consecutive perfect squares is odd.
\end{theorem}

\begin{proof}
  Recall the definition of an odd number to be any number that can be written $2n + 1$.
  Recall also the definition of a perfect square is any number that can be written $n^2$.\\
  Then,
   \begin{align*} 
     (n+1)^2 - (n^2)\\
     n^2+n+n+1-n^2\\
     2n + 1\\
   \end{align*} 
\end{proof}

\begin{other}{Result}
  Thus, the difference between any two consecutive perfect squares is always odd since it can always be written $2n+1$.
\end{other}

\begin{theorem}{17}
  At least 2 of 1,185 daily flights at Chicago O'Hare International Airport must take off within 90 seconds of each other
\end{theorem}

\begin{proof}
  Recall that the pigeonhole principle states that if $k$ items are put into $n$ containers,
  and $k > n$, then at least one container must have more than one item.\\
  Let $k=$ flights per day and $n=$ 90 second time intervals\\
  Then,
 \begin{align*}
  24 \cdot 60 \cdot 60 &= 86,400\text{ seconds per day}\\
  n = 86,400/90 &= 960\\
  k = 1,185
 \end{align*}
 Since $1,185 > 960$ at least one 90 second interval must contain more than one flight.
\end{proof}

\begin{other}{Result}
  Thus, per the pigeonhole principle, there must be at least 2 flights that take off within 90 seconds of each other
\end{other}

\begin{theorem}{22a}
  In a list of five digit distinct numbers of length 20 there are no two subsets that share the same sum.
\end{theorem}

\begin{proof}
  Recall that the pigeonhole principle states that if $k$ items are put into $n$ containers,
  and $k > n$, then at least one container must have more than one item.\\
  Let $k=$ subsets and $n=$ range of possible sums\\
  Then,
  \begin{align*}
    k&= 2^{20} - 1\\
     &=1,048,575\\
    n&= 20 \cdot99,999\\
     &=1,999,980\\
  \end{align*}
\end{proof}

\begin{other}{Result}
  Thus, through pigeonhole principle, there is no guarantee that there are two subsets with the same sum in a list of five distinct numbers of length 20.
\end{other}

\begin{theorem}{22b}
  The smallest list of five digit numbers in which two subsets
  have the same sum are of length 22.
\end{theorem}

\begin{proof}
  Recall that the pigeonhole principle states that if $k$ items are put into $n$ containers,
  and $k > n$, then at least one container must have more than one item.\\
  Let $k=$ subsets and $n=$ range of possible sums\\
  For a list of length 22, we calculate,
  \begin{align*}
    k &= 2^{22} -1\\
      &= 4,194,303\\ 
    n &= 22 \cdot{99,999}\\
      &=2,199,978
  \end{align*}
  The pigeonhole principle guarantees at least two subsets share the same sum in this case.
  For a list of length 21, we calculate,
  \begin{align*}
    k &= 2^{21} -1\\
      &= 2,097,151\\
    n &= 21 \cdot{99,999}\\
      &=2,099,979
  \end{align*}
  The pigeonhole principle fails to guarantee two subsets share the same sum in this case.
\end{proof}

\begin{other}{Result}
  Hence, $n = 22$ is the smallest integer for which the number of subsets must
  exceed the range of possible sums, ensuring at least two subsets share a common sum.
\end{other}

\begin{theorem}{23}%Suggestion: do one case for n odd and one case for n even.
  If n is any integer, then $3n^3 + n + 5$ is odd.
\end{theorem}

\begin{proof}
  Recall the definition of an odd number to be a number, $n$, that can be written as $2k+1$ where $k$ is an integer.\\
  Then,
  \begin{align*}
    3n^3 + n + 5 &= 2(2k+1)^3+(2k +1) + 5\\
                 &=3(8k^3+12k^2+6k+1) +2k +1+5\\
                 &=24k^3 +36k^2+18k+3+2k+6\\
                 &=24k^3+36k^2+20k+9\\
                 &=2(12k^3 +18k^2+10k+4)+1\\
  \end{align*}
  Let $m=12k^3+18k^2+10k+4$. Thus, the expression translates to $2m+1$, which is odd.\\
  \\
  Recall the definition of an even number to be a number, $n$, that can be written as $2k$ where $k$ is an integer.\\
  Then,
  \begin{align*}
    3(2k)^3+2k+5\\
    3(8k^3)+2k+5\\
    24k^3+2k+5\\
    2(12k^3+k+2)+1\\
  \end{align*}
  Let $m=12k^3+k+2$. By substitution, we get $2m+1$, which is odd.
\end{proof}

\begin{other}{Result}
  Therefore, through substitution of either an odd or even integer,
  we find the expression always takes the form $2m +1$.
\end{other}
% --------------------------------------------------------------
%     You don't have to mess with anything below this line.
% --------------------------------------------------------------
 
\end{document}
