% --------------------------------------------------------------
% This is all preamble stuff that you don't have to worry about.
% Head down to where it says "Start here"
% --------------------------------------------------------------
 
\documentclass[12pt]{article}
 
\usepackage[margin=1in]{geometry} 
\usepackage{amsmath,amsthm,amssymb}
 
\newcommand{\N}{\mathbb{N}}
\newcommand{\Z}{\mathbb{Z}}
\newcommand{\W}{\mathbb{W}}
 
\newenvironment{theorem}[2][Theorem]{\begin{trivlist}
\item[\hskip \labelsep {\bfseries #1}\hskip \labelsep {\bfseries #2.}]}{\end{trivlist}}
\newenvironment{lemma}[2][Lemma]{\begin{trivlist}
\item[\hskip \labelsep {\bfseries #1}\hskip \labelsep {\bfseries #2.}]}{\end{trivlist}}
\newenvironment{exercise}[2][Exercise]{\begin{trivlist}
\item[\hskip \labelsep {\bfseries #1}\hskip \labelsep {\bfseries #2.}]}{\end{trivlist}}
\newenvironment{reflection}[2][Reflection]{\begin{trivlist}
\item[\hskip \labelsep {\bfseries #1}\hskip \labelsep {\bfseries #2.}]}{\end{trivlist}}
\newenvironment{proposition}[2][Proposition]{\begin{trivlist}
\item[\hskip \labelsep {\bfseries #1}\hskip \labelsep {\bfseries #2.}]}{\end{trivlist}}
\newenvironment{corollary}[2][Corollary]{\begin{trivlist}
\item[\hskip \labelsep {\bfseries #1}\hskip \labelsep {\bfseries #2.}]}{\end{trivlist}}
\newenvironment{homework}[2][Homework]{\begin{trivlist}
\item[\hskip \labelsep {\bfseries #1}\hskip \labelsep {\bfseries #2.}]}{\end{trivlist}}
\newenvironment{other}[2][]{\begin{trivlist}
\item[\hskip \labelsep {\bfseries #1}\hskip \labelsep {\bfseries #2.}]}{\end{trivlist}}
 
\begin{document}
 
% --------------------------------------------------------------
%                         Start here
% --------------------------------------------------------------
 
%\renewcommand{\qedsymbol}{\filledbox}
 
\title{Homework Template}%replace X with the appropriate homework chapter or title
\author{Your Name\\ %replace with your name
MAT 203 - Discrete Mathematics} 
 
\maketitle
 
\begin{theorem}{x.yz} %You can use theorem, proposition, homework, exercise, or reflection here.  Modify x.yz to be whatever number you are proving or solving
Delete this text and write theorem statement here.
\end{theorem}
 
\begin{proof}
Blah, blah, blah.  Here is an example of the \texttt{align} environment: %Note 1: The * tells LaTeX not to number the lines.  If you remove the *, be sure to remove it below, too.
%Note 2: Inside the align environment, you do not want to use $-signs.  The reason for this is that this is already a math environment. This is why we have to include \text{} around any text inside the align environment.
\begin{align*}
\sum_{i=1}^{k+1}i & = \left(\sum_{i=1}^{k}i\right) +(k+1)\\ 
& = \frac{k(k+1)}{2}+k+1 & (\text{by inductive hypothesis})\\
& = \frac{k(k+1)+2(k+1)}{2}\\
& = \frac{(k+1)(k+2)}{2}\\
& = \frac{(k+1)((k+1)+1)}{2}.
\end{align*}
\end{proof}
 
\begin{proposition}{x.yz}
Let $n\in \Z$.  
\end{proposition}
 
\begin{proof}[Disproof]%Whatever you put in the square brackets will be the label for the block of text to follow in the proof environment.
Blah, blah, blah.  I'm so smart.
\end{proof}
 
 \vspace{2 in}
 
\begin{other}{Summations} Working with summation notation
\newline

Inline: $\sum_{n=1}^{\infty} 2^{-n}$ 

\vspace{.5 in}

Display style:\\
$$\sum_{n=1}^{\infty} 2^{-n}$$
\end{other}

\begin{other}{Set Operations and Relations} How to write set notation
\newline

$\emptyset$, $\N$, $\Z$, $\W$

$A\cap B$  

$A\cup B$  

$\neg{A}$ 

$\bar{A}$  

$\overline{A}$ 

$\sim{A}$

$\overline{A \cup B}$

$$\overline{A \cup B}$$
$A - B$,   $A\backslash B$\\
$A \subset B$,   $A \subseteq B$\\
$A \supset B$,   $A \supseteq B$\\
$a \in A$,   $a \notin A$\\
$A = B$,  $A \neq B$\\
$A \times B$\\
$A = \{1,2,3,4,5 \}$, $B = \{(x,y)|x<1 \wedge y\geq -2 \}$, $C = \{ 3,6,9,\ldots,6000\}$\\
$n! = n(n-1)(n-2)\cdots 3 \cdot 2 \cdot 1$\\
\end{other}
\pagebreak
\begin{other}{Logic Symbols and Operators} Using formal logic symbols


$p \wedge q$, $p \vee q$

$\sim p$, $\neg p$

$p \Rightarrow q$, $p \rightarrow q$ 

$p \Leftrightarrow q$, $p \leftrightarrow q$

$\forall$, $\exists$

$\therefore$

\end{other}

\begin{other}{Number Theory and Miscellaneous} Assorted symbols

$a \equiv b \pmod{c}$

Greek letters: $\alpha$, $\Omega$, $\delta$, $\Delta$, etc.

\end{other}


% --------------------------------------------------------------
%     You don't have to mess with anything below this line.
% --------------------------------------------------------------
 
\end{document}
